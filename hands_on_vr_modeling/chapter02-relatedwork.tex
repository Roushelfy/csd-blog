\section{Related Work}

% Our work builds on three complementary areas: \textit{virtual clay modeling}, \textit{physics-aware interactions in VR}, and \textit{3D modeling tools in VR}. While virtual clay modeling captures the tactile qualities of working with deformable materials, its integration into immersive VR environments remains limited. Physics-aware interactions in VR provide realism but often focus on rigid objects or interaction fidelity rather than intuitive shape editing. Existing VR modeling tools offer powerful geometric techniques but require significant expertise, limiting accessibility. By bridging these gaps, we introduce a physics-based hands-on modeling framework that combines the realism of physics-aware interactions with the accessibility and creativity of intuitive VR tools.

\paragraph{Virtual Clay Modeling}
Virtual modeling of clay-like materials using dexterous input offers significant advantages over traditional 3D modeling, including high expressivity and a lower learning curve, effectively replicating the tactile experience of working with physical clay~\cite{sheng2006interface, chatterjee2024free}. Despite these benefits, the computational demands of simulating dexterous hand-object interactions have limited the integration of virtual clay modeling into VR environments, with most existing implementations constrained to 2D screens. Early approaches to virtual clay modeling utilized dynamic subdivision solids~\cite{mcdonnell2001virtual} and sculpting tools~\cite{galyean1991sculpting}, followed by the application of plasticity models to capture essential physical properties of real clay, such as mass conservation and surface tension effects~\cite{dewaele2004interactive}. To improve interactivity, \citet{barreiro2021natural} introduced a particle-based viscoplasticity model with ultrasound haptic feedback, although the fidelity of finger-based interactions remained insufficient for detailed 3D modeling tasks. Additionally, physical proxies have been employed to enhance the virtual clay experience~\cite{sheng2006interface, pihuit2008hands, marner2010augmented}, such as combining direct finger input with deformable physical devices to enable clay-like sculpting operations, including deforming, smoothing, pasting, and extruding~\cite{sheng2006interface}. Efforts to adapt virtual clay modeling to VR environments have been limited. For instance, \citet{jang2014airsculpt} and \citet{moo2021virtual} made preliminary attempts to integrate these techniques into VR; however, their approach faced challenges in achieving the robustness and accuracy needed for intricate hand-object interactions.
To overcome these challenges, our work introduces an intuitive, contact-based VR modeling framework leveraging MPM \cite{jiang2016material} and 3D GS \cite{kerbl20233d}. Our approach significantly enhances real-time performance and hand-object interaction accuracy, enabling a more natural and immersive experience that underscores the unique benefits and potential of clay-like modeling in virtual environments.


% In the virtual clay process, providing passive tactile feedback for hands has been a topic of significant interest~\cite{mcdonnell2001virtual, pihuit2008hands, barreiro2021natural}.


\paragraph{Physics-Aware Interactions in VR}
Physics-aware hand-object interactions in VR have been extensively studied, enabling users to engage with various virtual phenomena such as particle-based animations~\cite{arora2019magicalhands}, deformable objects~\cite{moo2021virtual, deng2023phyvr, jiang2024vr}, and fluids~\cite{eroglu2018fluid}. These approaches often utilize hand-tracking data to map human hand motions to rigged virtual hand models, enabling dynamic interactions in immersive environments. For example, \citet{kumar2015mujoco, lougiakis2024comparing, holl2018efficient} use tracked hand information to define the pose of the virtual hand in each frame while incorporating frictional contact, achieving realistic hand-rigid-body interactions. Building on this, \citet{verschoor2018soft, jacobs2011soft, smith2020constraining} simulate soft hands using nonlinear soft tissue models, which allow for more natural and realistic interactions.
While these works primarily focus on interactions between hands and rigid objects, research on hand-deformable-object interactions has also gained traction. For instance, \citet{deng2023phyvr} developed PhyVR, a unified particle system for freehand interactions with multiple virtual materials; VR-GS~\cite{jiang2024vr} facilitates handle-based interaction with physically simulated elastic objects represented by 3D Gaussian Splatting~\cite{kerbl20233d}. However, these efforts largely emphasize physics-aware interactions rather than shape editing, which holds significant promise for creative tasks such as animation control~\cite{arora2019magicalhands}. To address this gap, we focus on physics-based hands-on 3D modeling in VR, leveraging both contact- and gesture-driven interactions to enable intuitive creation and editing of deformable objects for creative applications.



\paragraph{3D Modeling Tools in VR}
Creative tools for 3D modeling in VR encompass different categories. Among drawing-based modeling tools, Surface Drawing~\cite{schkolne2001surface} stands as a pioneering approach, enabling users to draw strokes in space using hand motions and edit 3D models with tangible tools. To enable smoother 3D curve creation, Dynamic Dragging~\cite{keefe2008tech} introduces an adaptive drag line that adjusts based on curvature and drawing speed. SurfaceBrush~\cite{rosales2019surfacebrush} and AdaptiBrush~\cite{rosales2021adaptibrush} convert dense collections of artist-drawn stroke ribbons into user-intended manifold free-form 3D surfaces. Cassie~\cite{yu2021cassie} is a conceptual modeling system in VR that leverages free-hand mid-air sketching and a novel 3D optimization framework to create a connected curve network. In addition, inspired by FiberMesh~\cite{nealen2007fibermesh}, RodMesh~\cite{schulz2019rodmesh} replaces the curve drawing with two-handed bending and stretching of virtual rods, allowing users to define outline shapes that are subsequently inflated into manifold mesh surfaces. \citet{peng2020autocomplete} presents a keyframe-based animated sculpting system that autocompletes user edits through an intuitive brushing interface. Commercial tools, such as Shapelab, Adobe Substance 3D Modeler\footnote{\url{https://www.adobe.com/products/substance3d/apps/modeler.html}}, and Kodon\footnote{\url{https://www.kodon.xyz/}}, mainly focus on geometric editing through polygonal or voxel-based sculpting. These tools allow users to manipulate geometry regions and apply operations with dynamic topology updates. Notably, Gravity Sketch introduces subdivision modeling, enabling users to create complex meshes from simple base topology while maintaining smoothness. Inspired by these prior works, we aim to further reduce the demand for professional 3D modeling expertise with physics-aware hands-on shape modeling, making 3D modeling in VR more accessible to novice users.

% However, such modeling approaches in VR typically require professional 3D modeling skills. In contrast, our physics-based shape-modeling method provides more intuitiveness and efficiency making 3D modeling more accessible for novice users.

% Radiance Field-based editing has also been explored with semantics-controlled large scene segmentation in VR~\cite{schieber2024semantics}. SCGS~\cite{schieber2024semantics} allows scene editing and the extraction of scene parts for VR. \todo{are the following works also radiance field based? if not, should connect the transit of topics.} Apart from radiance filed based methods,

% \todo{The related works are comprehensive, and the discussion of each work is generally adequate. Can consider providing a more clear grouping of the works, and then try to more smoothly connect the groups when introducing}


% SymbiosisSketch- Combining 2D & 3D Sketching for Designing Detailed 3D Objects in Situ
% ScafoldSketch- Accurate Industrial Design Drawing in VR