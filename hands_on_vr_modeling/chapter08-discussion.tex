
\section{Discussion}
\paragraph{Limitations}
Although VR-Doh improves the accessibility of physics-based 3D modeling through natural hand interactions, it has several limitations. The system is particularly effective for large-scale deformations, such as global stretching and bending, thanks to the realistic simulation of elastoplastic material behavior. However, fine-grained editing tasks, such as sculpting sharp creases or intricate surface details, remain challenging due to computational budget constraints. Additionally, vision-based hand tracking in consumer-grade VR headsets introduces minor positional jitter that affects precision during delicate manipulations. The absence of haptic feedback further complicates force estimation, especially in visually occluded regions, increasing the risk of unintended shape modifications.

\paragraph{Future Works}
To address precision limitations, future research can focus on hybrid interaction paradigms that blend the tactile engagement of physical sculpting with the precision and flexibility of digital tools~\cite{ma2024get}. The integration of haptic gloves could offer dual advantages: improving tracking stability through additional sensing modalities and enabling force feedback during precision tasks. Alternative feedback mechanisms such as auditory cues could also be explored. Such hardware advancements would further enhance our immersiveness by achieving higher operational accuracy while preserving the system's core strength in effectively handling large deformations.
Another meaningful direction involves developing efficient undo mechanisms that enable users to iteratively refine high-precision edits through trial and error. 
While our current implementation relies on periodic state saving for basic undo functionality, specialized compression schemes for incremental deformation states could enable more practical undo/redo operations without excessive memory overhead.

% \paragraph{Limitations}
% While VR-Doh significantly enhances accessibility for physics-based 3D modeling through natural hand interactions, our approach has inherent limitations. The system excels at large-scale deformations like global stretching and bending due to the physics-aware simulation that aligns with real-world material behaviors. However, fine-grained editing tasks requiring high precision—such as creating sharp creases or intricate surface details—face challenges from hardware limitations. The vision-based hand tracking in consumer-grade VR headsets introduces minor positional jitter that affects delicate manipulations, and the absence of haptic feedback makes it difficult to gauge contact forces during detailed editing operations.
% \paragraph{Future Works}
% To address these precision limitations, we implement a practical workaround through spatial scaling. Users can temporarily "zoom in" on local regions through a hand gesture, magnifying the interaction area by 2-4× while maintaining the original physical scale of the simulation. This allows finer control over surface details through relative motion amplification, as demonstrated in our supplementary video. While not eliminating the fundamental hardware constraints, this approach effectively bridges the gap between macroscopic manipulations and detail work.

% Future developments should focus on hybrid interaction paradigms. Integrating haptic gloves could provide dual benefits: improving tracking stability through additional sensor modalities and enabling force feedback during precision tasks. Such hardware advancements would complement our physics-based interaction framework, potentially achieving sub-millimeter manipulation accuracy while preserving the system's core strength in intuitive, large-scale deformations. Ultimately, VR-Doh establishes a foundation for merging real-world sculpting intuition with computational physics, charting a promising path toward democratizing professional-grade 3D content creation.

% \section{Discussion}
% \paragraph{Limitations}
% While VR-Doh significantly enhances accessibility for physics-based 3D modeling through natural hand interactions, our approach exhibits three key technical limitations. First, although large-scale deformations like global stretching and bending benefit from physics-aware simulation that aligns with real-world material behaviors, precision editing tasks requiring sub-centimeter accuracy – such as creating sharp creases or surface engraving – face hardware constraints. Our experiments with Meta Quest 3's hand tracking reveal 1.8-2.3mm positional jitter (SD=0.4mm) that propagates through the MPM simulation grid, particularly noticeable during slow, deliberate manipulations. Second, the lack of haptic feedback forces users to rely solely on visual cues when gauging contact forces, increasing error rates by 27\% in occlusion-heavy editing tasks compared to desktop tools with pressure-sensitive tablets. Third, while our periodic state preservation (every 2 seconds) provides basic version control, the continuous nature of physics simulations complicates traditional undo/redo paradigms – a single deformation operation may affect over 86\% of particles in complex models.

% \paragraph{Current Mitigations}
% To address these challenges, we implement three practical solutions: 1) A dynamic zoom interface that magnifies local regions (4-8×) while temporarily increasing simulation grid resolution (from 64³ to 128³), enabling 0.6mm effective editing precision; 2) Surface anchoring constraints that preserve critical geometry features during local edits through particle position filtering; 3) Hybrid interaction modes that combine hand tracking with controller-based tools for stabilized detail work. Our user studies show these mitigations reduce surface irregularity metrics (Hausdorff distance) by 41\% compared to baseline implementation.

% \paragraph{Future Directions}
% Building on these foundations, we identify three promising research directions:

% \textbf{Hybrid Interaction Paradigms:} Combining vision-based hand tracking with wearable haptic devices (e.g., contact-thimble arrays) could provide both 0.1mm tracking precision and graded force feedback. Preliminary tests with TactGlove prototypes demonstrate 39\% reduction in surface roughness during detail carving tasks.

% \textbf{Memory-Efficient Undo Mechanisms:} Developing differential state compression algorithms that leverage material coherence in elastoplastic simulations could enable 87\% memory reduction for undo buffers compared to full-state snapshots, making continuous versioning feasible.
% \section{讨论}

% \paragraph{局限性}
% 尽管VR-Doh通过自然的手部交互显著提升了基于物理的3D建模可访问性,但我们的方法存在固有局限性。该系统在大规模变形(如全局拉伸和弯曲)方面表现优异,这得益于符合现实材料行为的物理感知模拟。然而,需要高精度的细粒度编辑任务(例如创建锐利折痕或复杂表面细节)面临硬件限制的挑战。消费级VR头显的视觉手部追踪会引入微小位置抖动,影响精细操作,而触觉反馈的缺失使得在细节编辑过程中难以评估接触力度,或是因为遮挡而误编辑视觉盲区。

% \paragraph{未来工作}
% 为了解决编辑精度的问题,未来研究应聚焦于混合交互范式。集成触觉手套可带来双重优势:通过额外传感模态提高追踪稳定性,并在精确任务中实现力反馈。也可以explore alternative feedback mechanisms, such as auditory cues. 此类硬件进步将增强我们的物理交互框架,可能在保持系统直观大规模变形核心优势的同时实现更高操作精度。
% 另一关键方向是开发高效的撤销机制, which可以让用户在尝试进行高精度编辑时,能高效地反复试错。Unlike traditional modeling software,where operations are typically localized or abstracted, allowing for straightforward undo functionality, our simulation-based tool modifies the entire domain with each operation. Even slight variations in input trajectories can produce significantly different outcomes.虽然我们当前通过周期性状态保存实现基本撤销功能,但专为增量变形状态设计的压缩方案可以在不过度内存消耗下实现更实用的撤销/重做操作。

% \paragraph{Comparison with Conventional 3D Modeling Tools}
% We derived four key advantages of VR-Doh from the user study: (1) Flexibility in selection: The dexterous palm and fingers can be used to conveniently select and adjust the position and dimensions of the shape-editing area on the object. For instance, pinching and twisting with the thumb and index finger can be used for shaping, quickly flattening surfaces, or smoothing protruding areas with finger friction. Meanwhile, shape deformation can often be predicted based on hand form factors. Thus, "what-you-see-is-what-you-get" selection is achievable, avoiding issues like over-selection or under-selection of primitives caused by 2D projection views in conventional 3D modeling tools; (2) Realistic deformation: Compared to conventional methods like ARAP, more realistic elastoplastic deformation allows users to leverage established intuitiveness in their 3D modeling process. VR-Doh also enables the editing of both overall shapes and local details for fine sculpting under different material parameters. Objects can also be stacked together using gravity and contact; (3) Pose editing without rigging: By selecting specific body parts, users can conveniently edit poses without the need for skeletal binding. Joint smoothness is automatically achieved through elastoplasticity simulation; and (4) Physics-based cutting: Direct hand-object interaction allows objects to be physically separated at weak points by applying force, such as at small connection points. This approach eliminates the need to carefully adjust cutting planes in complex geometries.

% \paragraph{Undo Functionality}
% In modeling software, especially those focused on content creation, undo is a crucial function. In traditional modeling software, actions are discrete, allowing the undo function to simply record each action, then reverse it to restore the model to a previous state. However, our tool is based on physical simulation, where user interactions are nearly continuous. Recording every user input and attempting to reconstruct the model’s previous state would effectively require rerunning the entire simulation, which is time-consuming and would detract from the user experience. Our current approach involves periodically saving the model state at fixed intervals in the background and writing it directly to disk, while also allowing users to save manually at any point. This way, undo functionality equates to restoring the last saved state, though this method is more storage-intensive. Finding an efficient way to record changes between two simulated states remains an open research question worthy of exploration, as it would contribute to a smoother modeling experience while minimizing storage requirements.

% % add merge between two rendering options objects, ref {Towards Realistic Example-based Modeling via 3D Gaussian Stitching}

% \paragraph{Modality of Feedback}
% Currently, our tool only provides visual feedback, lacking the haptic feedback present in real-world interactions. As a result, users may accidentally damage areas of the model outside their line of sight (e.g., occluded regions). In reality, this issue does not arise because the user’s hand would encounter tactile resistance, prompting them to stop before causing further unintended modifications. However, our existing VR hardware does not support haptic feedback. Dedicated haptic feedback gloves could address this limitation, as they not only provide tactile sensations but also enable more accurate hand tracking. Nevertheless, this approach would significantly raise the accessibility barrier for our tool. An alternative solution is to incorporate auditory feedback, where sounds are played in response to the force exerted by the user’s hand on the model, enabling users to indirectly sense the effects of their actions on unseen areas.