\section{System Framework}
\todo{can move this section to be the first paragraph of the next section}
\begin{figure*}[ht]
    \centering
    \includegraphics[width=\linewidth]{img/System.pdf}
    \caption{System Framework}
    \label{fig:system}
\end{figure*}

\begin{algorithm}
\caption{Overall Framework}
\begin{algorithmic}[1]
\While{True}
    \State Update Input()
    \For{$i \gets 1$ to substeps}
        \State substep()
    \EndFor
    \State Update Rendering()
\EndWhile
\end{algorithmic}
\end{algorithm}
Our system integrates real-time simulation, high-quality rendering, and  flexible user interaction, supporting intuitive VR modeling. The overall system framework is shown in \autoref{fig:system}.  We use the Material Point Method (MPM) \cite{hu2018moving} for simulation, which accurately handles elastoplastic materials and large deformations. For rendering, the system offers two options: Gaussian splatting \cite{kerbl20233d}, which is used to refine reconstructed or generated 3D models, and mesh rendering, which uses the Marching Cubes algorithm \cite{lorensen1998marching} to reconstruct a mesh from the particles, allowing users to create new models from scratch. User interactions include direct hand interaction, tool-based interaction, and gesture inputs.