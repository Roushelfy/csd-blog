\section{Conclusion}
We presented VR-Doh, a VR-based 3D modeling system that integrates advanced elastoplastic simulation with intuitive, hand-based interaction. Leveraging the Material Point Method (MPM) and incorporating innovations such as localized simulation, particle-level collision handling, and decoupled appearance and physical representations, the system achieves real-time responsiveness while delivering high-quality simulation and rendering. These technical innovations make VR-Doh accessible to novice users while empowering experts to perform detailed and expressive modeling. Extensive evaluations and user studies showcased the system's ability to provide an intuitive and immersive modeling experience. Compared to traditional tools, VR-Doh offers a more natural and accessible approach, with user-identified advantages such as flexible selection, realistic deformation, rig-free pose editing, and physics-based cutting, paving the way for a new paradigm in 3D modeling.

%  Compared to traditional modeling tools, VR-Doh offers a more natural and accessible approach in specific scenarios, with four key advantages identified from the user study:\todo{consider moving it to user study section, and then polish the above paragraph}
% \textit{1) Flexible selection:} Users can intuitively select and adjust editing areas on an object using dexterous hand movements. For example, pinching and twisting with the thumb and index finger can flatten surfaces or smooth protrusions, while shape deformation often aligns with hand gestures, enabling "what-you-see-is-what-you-get" precision. This also avoids over-selection or under-selection issues common in 2D-projected views of conventional tools.
% \textit{2) Realistic deformation:} Unlike conventional methods like ARAP, VR-Doh uses elastoplastic deformation, offering more realistic results. Users can edit overall shapes and fine details, stack objects using gravity and contact, and leverage material-specific properties for intuitive sculpting.
% \textit{3) Pose editing without rigging:} Specific body parts can be selected and adjusted directly, allowing pose edits without the need for skeletal rigging. Joint smoothness is automatically handled by the elastoplastic simulation.
% \textit{4) Physics-based cutting:} Objects can be separated at weak points by applying physical forces, such as breaking small connections, without the need to manually adjust cutting planes in complex geometries as in traditional 3D modeling software.

\paragraph{Discussion and Future Works}
Our method offers several promising directions for future research. Unlike traditional modeling software, where operations are typically localized or abstracted, allowing for straightforward undo functionality, our simulation-based tool modifies the entire domain with each operation. Even slight variations in input trajectories can produce significantly different outcomes. While our current solution relies on periodically saving states, developing a memory-efficient undo mechanism would be highly beneficial. Additionally, the absence of haptic feedback in our system makes it challenging for users to avoid unintended modifications, especially in occluded regions. Improving the user experience could involve incorporating advanced haptic devices or exploring alternative feedback mechanisms, such as auditory cues, while ensuring accessibility remains a priority.