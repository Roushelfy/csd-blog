\section{Introduction}
% \cmy{We can emphasize the novelty of physics-based modeling with regard to geometry-based modeling. The latter is widely seen in commercial tools but often requires careful observation of real-life objects and specialized expertise to get familiar with the tools, and to learn how to fine-tune and adjust.
% In contrast, physics-based modeling aligns more closely with human intuition and natural interactions:
% 1. Provides a wide range of tools that users can easily use without learning.
% 2. Users can predict the results of their actions very well according to their daily experience, reducing the need for extensive professional experience.
% 3. Offers modeling potential on par with geometry-based approaches, as evidenced by the diversity of the real world.}

With the growing demand for digital content, there is an urgent need to address key challenges in 3D content creation: ease of use, scalability, efficiency, and quality. Traditional 3D modeling tools impose significant barriers for novice users, requiring expertise in real-world observation, specialized techniques, and iterative fine-tuning. This reliance on professional skills limits accessibility, hindering wider participation in high-quality content creation.
% As the demand for digital content continues to grow, there is an increasing need for ease of use, greater quantity, improved efficiency, and higher quality in 3D content creation. Traditional 3D modeling tools impose significant barriers to entry for novice users. These tools often require careful observation of real-world objects, mastery of specialized techniques, and iterative fine-tuning to achieve the desired outcome. This reliance on professional expertise limits accessibility for broader audiences in high-quality content creation.

Meanwhile, 3D modeling in Virtual Reality (VR) has become increasingly popular due to its ability to provide immersion and depth, thereby enhancing designers' creativity and collaboration~\cite{rosales2019surfacebrush, yu2021cassie}. Commercial tools like Shapelab~\cite{shape-lab-no-date}
% \footnote{\url{https://shapelabvr.com/}} \footnote{\url{https://gravitysketch.com/}} 
and Gravity Sketch~\cite{gravity-sketch-no-date} have advanced this vision. However, their approaches are still limited to procedural methods, such as drawing curves to form surfaces from scratch, which are restricted to geometric techniques and require significant skills. 

In contrast, physics-aware shape modeling introduces a novel paradigm that aligns more closely with human intuition and natural interactions. In this approach, objects are simulated as deformable solids, reshaped through contact forces or boundary conditions specified by the user~\cite{Fang2021IDP}. In the context of VR, using hands as an input modality is particularly compelling, as it mimics the way users naturally manipulate physical objects, such as clay. This approach leverages users' prior experience with creating real-world objects, making the modeling process more intuitive and accessible~\cite{schulz2019rodmesh, arora2019magicalhands, pihuit2008hands}.

Building on these foundational advantages, we integrate physics-based simulation into VR to deliver an intuitive and efficient hands-on 3D modeling system, \textit{VR-Doh}. Our primary objective is to enable users to create and edit virtual objects with a high degree of realism, leveraging direct hand-based contact and gestures alongside interactive feedback. To efficiently support large deformations and contact handling, we adopt the Material Point Method (MPM) \cite{jiang2016material, hu2018moving}, a hybrid Lagrangian-Eulerian framework capable of simulating versatile elastoplastic materials.

VR-Doh enables users to create 3D models by deforming and composing built-in primitive geometries, with surface meshes reconstructed from MPM particles. Additionally, based on \citet{xie2024physgaussian}, we extend support for editing existing 3D data represented by Gaussian Splatting (GS)~\cite{kerbl20233d}, allowing content creation through the modification of realistically rendered objects captured from the real world. To achieve real-time responsiveness despite the computational demands of simulation and rendering, we introduce optimizations such as particle-level collision handling and the decoupling of appearance and physical representations, allowing detailed hand-object interactions within budgeted simulation degrees of freedom.
Beyond hand-based contact and gestures, we integrate a suite of deformation tools driven by tracked hand motions, providing diverse and precise deformations to enhance users' creative flexibility.
To evaluate VR-Doh’s usability and effectiveness, we conducted user studies with participants of varying expertise. The study highlights its intuitive design, immersive feedback, and potential for complementing traditional modeling software in creative exploration. These findings offer insights into how users naturally leverage physics-based interactions, paving the way for future improvements and more immersive 3D modeling experiences in VR.

Our key contributions are summarized as follows:
\begin{itemize}
    \item \textit{A novel physics-based VR 3D modeling paradigm}: We present VR-Doh, a VR system that integrates physics-based simulations to deliver intuitive and immersive workflows, making 3D modeling more accessible to users of varying expertise.
    \item \textit{Intuitive modeling through natural interactions}: VR-Doh uses hand-based contact and gesture inputs together with deformation tools to support tasks from basic shape editing to complex modeling, mimicking real-world interactions.
    \item \textit{Real-time performance through technical innovations}: We ensure smooth and responsive interactions with optimizations such as localized simulation, decoupled appearance and physical representations, and particle-level collision handling for efficient computation.
    \item \textit{Comprehensive evaluation}: Extensive user studies and experiments validate VR-Doh's effectiveness, offering insights into how physics-based hand-object interactions enhance intuitive and creative 3D modeling in VR.
\end{itemize}

A key advantage of our approach is its ability to make 3D modeling in VR significantly more intuitive and accessible, especially for novice users. By incorporating physics-based deformations, VR-Doh provides a realistic and immersive modeling experience that surpasses traditional geometric modeling approaches. Hands-on input lowers the barrier to entry, enabling users to interact with virtual objects naturally and intuitively, while flexible selection of operation areas improves efficiency and enhances immersion. This direct interaction mimics real-world manipulation, allowing users to predict outcomes based on everyday experiences and reducing the reliance on specialized skills and technical expertise.

% Prior research has explored virtual clay interactions, but these have not yet been specifically implemented for use in VR~\cite{}. Despite prior research into physics-assisted modeling, existing systems often struggle with real-time responsiveness and user-friendly interactions.  (e.g., build a dumpling 3D model by basic volumetric primitives such as a sphere and an ellipsoid \cite{Fang2021IDP})

% Compared to 3D modeling with a mouse ...contact-based hands-on modeling... 
% 手相对于通过电脑鼠标建模的优势,即通过手指更加直观的选择理想的操作区域,并用过手的接触直接操纵物体的形状。而在使用鼠标建模的过程中,选择指定区域并且施加形变往往需要繁复的点击操作。当然这也导致无法精准的选择想要操作的区域,为解决此问题同时提供了mid-air手势操作区域的选择,通过施加force field使物体产生形变。这个过程最大的优势是降低了建模的学习成本,像真实世界clay过程一样操作虚拟物体。

% 3D modeling technology is a cornerstone in fields such as industrial design, film production, game development, and education. It empowers designers and developers to transform abstract concepts into tangible results, validate complex designs, and optimize production pipelines. However, traditional 3D modeling tools, typically geometry-based, impose significant barriers to entry for novice users. These tools often require careful observation of real-world objects, mastery of specialized techniques, and iterative fine-tuning to achieve the desired outcome. This reliance on professional expertise limits accessibility for broader audiences.

% With the advent of Virtual Reality (VR), 3D modeling has embraced a new dimension of creativity and interaction. VR’s immersive environments and natural spatial interfaces enhance the modeling experience, offering unprecedented levels of creative freedom and collaboration~\cite{liu2021systematic}. Commercial tools like ShapeLab and Gravity Sketch have explored VR modeling, yet most remain constrained by geometry-based approaches. These methods, while powerful, demand complex operations such as curve drawing and surface generation, which are challenging for users without prior expertise.

% \begin{itemize}
%     \item It provides a wide range of tools that users can easily use without prior training.
%     \item Users can predict the results of their actions based on everyday experiences, reducing reliance on professional knowledge.
%     \item It offers modeling capabilities comparable to geometry-based approaches, while enhancing immersion and expressiveness through physical interactions in VR.
% \end{itemize}