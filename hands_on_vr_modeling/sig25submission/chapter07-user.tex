% \section{User Study}
% To understand how experts and novices use VR-Doh for creative tasks in VR, we invited 6 participants (P1-P6) with 3-5 years of prior 3D modeling experience, and 6 participants without any prior experience (P7-P12). The study included two open-ended tasks to evaluate system usability, performance, user satisfaction, and to gather participants' feedback and suggestions.

% \subsection{Tasks \& Procedure}
% Each participant was required to conduct two tasks, with a 5-minute break in between: (1) \textit{Editing a 3D Model}: Participants were free to modify a pre-existing 3D model represented by GS, e.g., reshaping a plush toy or a cartoon character, to match their preferred design; (2) \textit{Creating a 3D Model from Scratch}: Participants could freely use basic shapes such as spheres, cubes, tori, and cylinders to create a new 3D model, such as a snowman or a hamburger.

% Before starting, participants practiced basic operations with a provided shape until they became familiar with the system. After completing the tasks, participants filled out the USE questionnaire~\cite{lund2001measuring} to evaluate usability and completed the Simulator Sickness Questionnaire (SSQ)~\cite{kennedy1993simulator} to measure any discomfort experienced during the study. Finally, participants took part in a 10-minute semi-structured interview to provide more in-depth insights on usability.

% \subsection{Results}
% Both expert and novice participants found VR-Doh to be highly intuitive, immersive, and efficient in their 3D modeling process, and they also claimed that the created shapes closely matched their initial design expectations. Some of their design outcomes are shown in \autoref{fig:user_study_results}. Participants rated the system highly on usefulness (5.6/7), ease of use (5.5/7), ease of learning (6.0/7), and overall satisfaction (5.9/7). Moreover, no discomfort and nausea were found. We also report their subjective feedback below.

% \paragraph{Intuition and Realism in 3D Modeling} Our participants spoke highly of the low learning curve of VR-Doh, allowing them to quickly engage in shape editing or create 3D models with their hands. P1 and P3 mentioned that directly manipulating the shape of objects with their hand was more flexible than deforming the object by setting control points. P5 found it more convenient to modify the shape without the need for rigging, which is tedious in conventional software. P1 stated that the rod tool was particularly useful and entertaining for poking holes and shaping details on a strawberry shape, and said it felt like \textit{"walking inside the strawberry."} The mid-air pinch interaction was highlighted, which allows users to select the area of the object they want to deform or sculpt. P2 considered that the ability to freely rotate the viewpoint in VR is more intuitive than the traditional three-view approach.

% In addition, our participants highly praised the modeling process for its realistic deformation and real-time rendering. As P4 mentioned, \textit{"Usually, I rarely look at the texture rendering while editing shapes because it’s very inefficient."} P2 described the process of editing the 3D model of the mermaid, saying, \textit{"It feels like the mermaid is really swimming in front of me."} P3 and P8 especially appreciated the feature to switch between different material types, which allowed them to edit the shape based on different deformation effects under varying plasticity types and material parameters. P9 favored the use of gravity during the modeling process to naturally stack objects together. Besides, the interactive feedback was another major advantage, allowing participants to immediately see the deformation results of their actions and make corresponding adjustments.

% \paragraph{Encountered Problems and Suggested Features}
% Our participants also identified a few areas for improvement. Hand-tracking instability was mentioned most frequently (P3, P6, P9, and P11), particularly when adjusting small or detailed parts of the object, resulting in unintended shape changes. Even though we have applied interpolation to the hand movements, this limitation is largely due to the constraints of the hand-tracking hardware. Therefore, participants expressed the need for an undo feature to help them recover from mistakes. Meanwhile, P2, P9, and P11 also suggested reducing the degrees of freedom for moving or rotating objects during the modeling process, similar to the approach used in pottery making. P1 and P7 also noted occasional performance issues in high-particle scenarios. For instance, P1 found that merging multiple objects together led to a loss of surface details. Besides, P1 and P8 suggested adding an operation to separate objects. P2 and P10 mentioned that objects would unintentionally collide with the simulation domain boundary during editing, and we have followed their suggestion of adding a one-click centering feature to avoid this issue.

% In summary, participants generally appreciated the system's efficiency in using hands, tools, or mid-air gestural input in their 3D modeling experience. Even expert users mentioned that for more detailed modeling scenarios, conventional modeling tools are still necessary. Nevertheless, VR-Doh is extremely well-suited for quickly shaping to express design concepts, supporting creative exploration, and facilitating communication. While there were some concerns regarding precision and the lack of an undo feature, this did not overshadow the tool's potential for creating high-quality 3D models for both novice and expert users.

\section{User Study}
To evaluate VR-Doh's usability, effectiveness, and potential applications, we conducted two user studies involving both novice and expert participants. The first study (\textit{US1}) focused on open-ended creative tasks to assess the system's intuitiveness, immersion, and efficiency in 3D modeling. The second study (\textit{US2}) directly compared VR-Doh with the conventional modeling software Blender, highlighting their respective strengths and limitations. We briefly summarize the process and findings of the two user studies here, with detailed information provided in the supplemental document.

\paragraph{US1: Usability and Effectiveness of VR-Doh}
US1 involved 12 participants, including 6 experts (P1-P6) with 3-5 years of 3D modeling experience and 6 novices (P7-P12) without prior experience, to evaluate VR-Doh's usability and performance. Participants were tasked with (1) editing a pre-existing 3D model (e.g., reshaping a plush toy) and (2) creating a 3D model from scratch (e.g., designing a snowman). After practicing basic operations, participants performed these tasks and completed the USE questionnaire~\cite{lund2001measuring} to evaluate usability and the Simulator Sickness Questionnaire (SSQ)~\cite{kennedy1993simulator} to assess discomfort. A 10-minute semi-structured interview was also conducted to gather detailed feedback.

Both expert and novice participants found VR-Doh intuitive, immersive, and efficient, with design outcomes closely matching their expectations (\autoref{fig:user_study_results}). The system received high ratings for usefulness (5.6/7), ease of use (5.5/7), ease of learning (6.0/7), and satisfaction (5.9/7), with no reports of discomfort or nausea. Participants praised the natural interaction modes, such as hand manipulation and mid-air gestures, for enabling realistic deformation and intuitive viewpoint control. Features like switching material types, leveraging gravity for stacking, and real-time rendering were particularly appreciated. However, limitations such as hand-tracking instability and the lack of an undo feature were noted, alongside suggestions for improved precision.

\paragraph{US2: Comparison with Blender}
US2 compared VR-Doh with conventional desktop-based modeling software, specifically Blender 4.3, involving six participants (four with 3-7 years of modeling experience and two without). Participants performed two goal-directed tasks—creating a Swiss roll and a donut—using both tools, guided by a step-by-step video tutorial. The order of tools was counterbalanced to minimize learning effects. On average, participants completed both tasks faster using VR-Doh and expressed a preference for its intuitive and natural interaction, particularly for the Swiss roll task.

Four key advantages of VR-Doh were identified: (1) flexible selection through dexterous hand movements, enabling "what-you-see-is-what-you-get" precision and avoiding over-/under-selection issues common in 2D-projected views; (2) realistic deformation leveraging elastoplastic simulation for intuitive editing of shapes and materials; (3) pose editing without rigging, where body parts can be adjusted directly with smooth joint transitions enabled by simulation; and (4) physics-based cutting, allowing objects to be physically separated at weak points without sophisticated manual adjustments of cutting planes. Participants found VR-Doh particularly effective for quickly shaping overall structures and realizing design ideas, although its precision in specialized tasks was lower than Blender due to real-time simulation constraints and the lack of haptic feedback. Overall, VR-Doh and Blender exhibited complementary strengths, with VR-Doh excelling in intuitive, global operations and Blender providing superior precision for detailed edits.


% \begin{figure}[H]
%     \centering
%     \includegraphics[width=\linewidth]{img/User_Study_Results.png}
%     \caption{\textbf{Designs from novice and expert participants.}}
%     \label{fig:user_study_results}
% \end{figure}

% \section{User Study 2: Comparison with Blender}
% To further investigate the differences between VR-Doh and conventional desktop-based modeling software, i.e., Blender, we invited six participants, including four with 3-7 years of modeling experience and two without, to perform two goal-directed 3D modeling tasks.

% \subsection{Tasks \& Procedure}
% Each participant was required to create both a Swiss roll and a donut using VR-Doh and Blender 4.3, guided by a step-by-step video tutorial. They were instructed to replicate the results demonstrated in the tutorial as closely as possible. When completing tasks with VR-Doh, participants watched the tutorial video until they claimed to have memorized the necessary operations. To minimize potential learning effects, the order of tools used was counterbalanced across participants. After completing the tasks, we conducted a semi-structured interview to invite participants to provide feedback and share their preferences regarding their modeling experiences with both tools.

% \subsection{Results}
% We found that participants, on average, required less time to complete the Swiss roll (X/X) and donut (X/X) tasks using VR-Doh compared to Blender. When participants were asked about their preferences for completing the tasks, X participants expressed a stronger preference for VR-Doh. Notably, one participant with modeling experience preferred using VR-Doh for the Swiss roll and Blender for the donut, indicating that VR-Doh may serve as a valuable complement to Blender in 3D modeling.

% \paragraph{Key Differences}
% Compared to Blender, we found VR-Doh provides a more natural modeling approach, making it easier for both experienced and novice participants to understand the operational steps for creating objects from the video. Four key advantages were identified in the user study:
% \textit{1) Flexible selection:} Users can intuitively select and adjust editing areas on an object using dexterous hand movements. For example, pinching and twisting with the thumb and index finger can flatten surfaces or smooth protrusions, while shape deformation often aligns with hand gestures, enabling "what-you-see-is-what-you-get" precision. This also avoids over-selection or under-selection issues common in 2D-projected views of conventional tools.
% \textit{2) Realistic deformation:} Unlike conventional methods like ARAP, VR-Doh uses elastoplastic deformation, offering more realistic results. Users can edit overall shapes and fine details, stack objects using gravity and contact, and leverage material-specific properties for intuitive sculpting.
% \textit{3) Pose editing without rigging:} Specific body parts can be selected and adjusted directly, allowing pose edits without the need for skeletal rigging. Joint smoothness is automatically handled by the elastoplastic simulation.
% \textit{4) Physics-based cutting:} Objects can be separated at weak points by applying physical forces, such as breaking small connections, without the need to manually adjust cutting planes in complex geometries as in traditional 3D modeling software.
% In summary, the above benefits of VR-Doh significantly enhance the efficiency of 3D modeling, enabling users to quickly figure out modeling steps and realize design ideas envisioned in the mind. As one participant noted, it helps avoid the common issue experienced with conventional modeling software, where "the hands cannot keep up with the mind."

% While the main disadvantage was reported by our participants, in more specialized modeling tasks, the precision of 3D models created using VR-Doh was found to be relatively lower than that of Blender. This is primarily due to the limited number of particles available during real-time simulation and rendering. Additionally, the lack of haptic feedback often led to over-adjustments. Overall, we found that VR-Doh is better suited for shaping the overall structure of objects due to its intuitive operation, whereas Blender excels in detailed, localized operations because of its precise selection of points, edges, and faces. Therefore, the two tools exhibit a complementary relationship.

% \textbf{Qualitative Insights:}
% Novice users highlighted the system's intuitive and accessible design, particularly praising the "pinch-and-sculpt" interaction as low-cost in terms of learning effort. The freedom to adjust parameters and the availability of real-time feedback were also seen as strengths. However, they identified several challenges: hand tracking inaccuracies occasionally disrupted operations, and the lack of geometric presets made starting new models less convenient. Performance limitations, especially in high-particle scenarios, were noted, with some users suggesting enhanced rendering and tracking precision. The absence of a multi-step undo feature was frequently mentioned as a limitation, with users expressing a need for better error correction during modeling. Despite these issues, most novices found the tool engaging and suitable for creative tasks, particularly appreciating its ability to simulate real-world interactions intuitively.

% \paragraph{Intuition and Realism in 3D Modeling}
% Both novice and expert users found the system to be intuitive and immersive, with the pinch-and-sculpt interaction being a key highlight. Novices appreciated the low learning curve, allowing them to quickly engage with the tool and start creating models without prior experience. The real-time feedback was another major strength, enabling users to immediately see the results of their actions and adjust accordingly. This real-time interaction helped foster creativity and made the modeling process feel more dynamic and responsive. Additionally, the system’s ability to simulate real-world interactions made the experience feel natural and engaging. Expert users also praised the system for its realistic deformation and real-time rendering, which provided a more lifelike experience compared to traditional modeling tools. The ability to manipulate models directly with hand-based input was seen as more intuitive than using a mouse, allowing for more direct and tactile control. Experts particularly appreciated the real-time material adjustments and flexible parameter controls, which enhanced their creative workflow and speed. %  This real-time interaction helped foster creativity and made the modeling process feel more dynamic and responsive. Additionally, the system’s ability to simulate real-world interactions made the experience feel natural and engaging.

% Our participants also identified a few areas for improvement. Hand-tracking inaccuracies were mentioned the most (P4, P9, and P11), particularly when adjusting small or detailed areas of a model. This limitation in precision is largely due to the constraints of the hardware used for hand tracking. Therefore, P3 and P4 suggested repair the object. Also, P2 and P9 suggested that we can fix one dimension of the object during the shape modeling process. Additionally, both groups noted occasional performance issues in high-particle scenarios. Another common concern was the absence of an undo feature, which made it difficult to recover from mistakes during the modeling process.