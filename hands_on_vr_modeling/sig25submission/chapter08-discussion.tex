
\section{Discussion}
\paragraph{Comparison with Conventional 3D Modeling Tools}
We derived four key advantages of VR-Doh from the user study: (1) Flexibility in selection: The dexterous palm and fingers can be used to conveniently select and adjust the position and dimensions of the shape-editing area on the object. For instance, pinching and twisting with the thumb and index finger can be used for shaping, quickly flattening surfaces, or smoothing protruding areas with finger friction. Meanwhile, shape deformation can often be predicted based on hand form factors. Thus, "what-you-see-is-what-you-get" selection is achievable, avoiding issues like over-selection or under-selection of primitives caused by 2D projection views in conventional 3D modeling tools; (2) Realistic deformation: Compared to conventional methods like ARAP, more realistic elastoplastic deformation allows users to leverage established intuitiveness in their 3D modeling process. VR-Doh also enables the editing of both overall shapes and local details for fine sculpting under different material parameters. Objects can also be stacked together using gravity and contact; (3) Pose editing without rigging: By selecting specific body parts, users can conveniently edit poses without the need for skeletal binding. Joint smoothness is automatically achieved through elastoplasticity simulation; and (4) Physics-based cutting: Direct hand-object interaction allows objects to be physically separated at weak points by applying force, such as at small connection points. This approach eliminates the need to carefully adjust cutting planes in complex geometries.

% \paragraph{Undo Functionality}
% In modeling software, especially those focused on content creation, undo is a crucial function. In traditional modeling software, actions are discrete, allowing the undo function to simply record each action, then reverse it to restore the model to a previous state. However, our tool is based on physical simulation, where user interactions are nearly continuous. Recording every user input and attempting to reconstruct the model’s previous state would effectively require rerunning the entire simulation, which is time-consuming and would detract from the user experience. Our current approach involves periodically saving the model state at fixed intervals in the background and writing it directly to disk, while also allowing users to save manually at any point. This way, undo functionality equates to restoring the last saved state, though this method is more storage-intensive. Finding an efficient way to record changes between two simulated states remains an open research question worthy of exploration, as it would contribute to a smoother modeling experience while minimizing storage requirements.

% % add merge between two rendering options objects, ref {Towards Realistic Example-based Modeling via 3D Gaussian Stitching}

% \paragraph{Modality of Feedback}
% Currently, our tool only provides visual feedback, lacking the haptic feedback present in real-world interactions. As a result, users may accidentally damage areas of the model outside their line of sight (e.g., occluded regions). In reality, this issue does not arise because the user’s hand would encounter tactile resistance, prompting them to stop before causing further unintended modifications. However, our existing VR hardware does not support haptic feedback. Dedicated haptic feedback gloves could address this limitation, as they not only provide tactile sensations but also enable more accurate hand tracking. Nevertheless, this approach would significantly raise the accessibility barrier for our tool. An alternative solution is to incorporate auditory feedback, where sounds are played in response to the force exerted by the user’s hand on the model, enabling users to indirectly sense the effects of their actions on unseen areas.

