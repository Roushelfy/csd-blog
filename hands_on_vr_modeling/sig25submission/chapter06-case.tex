\section{Case Study}
% All our results  provide a online 3D viewer on this website for the modeling results of this section: \href{https://anonymous.4open.science/w/VR-doh-85B8/}{Click to view 3D models}.

\paragraph{Featured Operations}
Figure~\ref{fig:Featured_Operations} highlights featured modeling operations in VR-Doh, including contact-based shape editing with hands and tools like slabs and rods, mid-air gesture-based bending and twisting, and the sourcing tool for adding new materials. The objects exhibit realistic elastoplastic deformations, closely resembling real-world behavior. This physical realism enables users to intuitively model objects by drawing on their real-life experiences.

% \begin{figure*}[hb]
%     \centering
%     \includegraphics[width=\linewidth]{img/baseOP.png}
%     \caption{Case of Basic operations}
%     \label{fig:Basic}
% \end{figure*}

\paragraph{Object Creation}
We demonstrate object creation using VR-Doh, as shown in \autoref{fig:mc_examples}. 
In the first row, a snowman is assembled, starting with a blank snowfield, molding the body, and adding details. 
The second row demonstrates the creation of a hamburger by stacking individual components to form a layered structure.
In the third row, a panda is carefully modeled from head to body, incorporating bamboo as an accessory that seamlessly integrates with its hands.
The fourth row depicts the crafting of a Swiss roll, where layers are rolled together to replicate a realistic pattern.
Finally, the fifth row shows the formation of steamed buns, shaped and arranged within a bamboo steamer.
These examples highlight VR-Doh’s strength in simplifying tasks that require precise spatial arrangements, which are significantly more challenging to achieve using a mouse on a 2D screen. With VR-Doh, object creation becomes as natural and intuitive as manipulating items in the real world.

\paragraph{3D GS Object Editing}
\autoref{fig:object_editing} demonstrates 8 examples of editing GS-represented objects, showcasing the intuitive and creative capabilities of VR-Doh. 
First, multiple mushroom-shaped houses were assembled, with their roofs sharpened and the mushrooms enlarged through hand manipulation and mid-air pinch gestures. The text "VR-Doh" was then sculpted onto the roof by hand. 
Next, a red pig transitioned from running to performing ballet through flexible selection and dragging of specific body parts, with its head rotated to face the audience using the pinch gesture. 
Two statues were brought to 'life': a terracotta figurine joyfully lifted its head and began drumming, while the Discobolus threw its discus with dynamic force. This involved using hands to cut and separate the discus before repositioning it. 
%
Moving on, wooden frames were interlocked and freely stacked into an arch through gravity and contact interactions, then merged seamlessly with greenery. The greenery’s complex shapes were also easily refined by hand. 
A girl’s posture was adjusted as her arms bent to mimic the act of eating a watermelon. 
Following that, flowers on a vase were flexibly selected using the pinch gesture, then intertwined in space with penetrations automatically avoided, demonstrating precise and convenient control. 
Finally, a capybara plush toy was made to appear fatter, while its head decoration was reshaped into a strawberry and placed in the front. These examples highlight VR-Doh’s ability to simplify complex modeling tasks while maintaining creativity and controllability. All examples are available on an anonymous online \href{https://anonymous.4open.science/w/VR-doh-85B8/}{3D viewer}.

% \subsection{Adjustments for Large Scenes}
% We take the classic scene of Gaussian splatting, "garden", as an example to demonstrate the adjustment capabilities of our tool for large-scale scenes. As shown in Figure \ref{fig:garden}, in the first row, we first deform the vase using direct hand contact and then insert a bouquet of roses into the vase by merging them together. Subsequently, we use hand gestures to bend the roses and repeatedly apply this process to three roses. In the second row, we move to the center of the garden scene, remove the original vase, deform the table using the tool, and then place the vase, which has been processed with roses, in the center of the table. The entire process took less than five minutes and closely resembled real-world interactions.
% \begin{figure*}[hb]
%     \centering
%     \includegraphics[width=\linewidth]{img/garden_new.png}
%     \caption{Case of Garden Adjustment}
%     \label{fig:garden}
% \end{figure*}

